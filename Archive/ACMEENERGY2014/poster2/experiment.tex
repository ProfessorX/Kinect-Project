\section{Experiment}

We carried out an experiment to corroborate the performance of our approach empirically. We gathered data from three vehicles of different classes --- SUV, sedan, and hatchback --- driven along the same path simultaneously. Our data collection apparatus consisted of Bluetooth ELM327 dongles plugged into the vehicles' onboard diagnostic (OBD) ports and paired with drivers' smartphones, on which were installed a mobile app we developed for collection and upload of OBD data from the vehicles, along with geolocation data, accelerometer readings and device identification from the smartphone.

Three different vehicles with three different drivers were used in this experiment: A) Ford Fusion 2012, 4 cylinder, 2.5 L. B) Hyundai Veloster 2014, 4 cylinder, 1.6 L and C) Lincoln MKX 2007, 6 cylinder, 3.7 L. We chose a 36.3 kilometer-long triangular circuit for the experiment, split up into three segments (routes) of lengths 10km. Each vehicle's driver was assigned a particular driving style: A) cautious, B) moderate, and C) aggressive. The data collection run consisted of two rounds of the circuit, which adds up to about 73 km. Hence, each route was covered twice.
