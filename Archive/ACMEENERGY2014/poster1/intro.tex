\section{Introduction} \label{sec:intro}

As automotive information systems read and process more data from vehicles and their environments, so too have increased the consumer expectations of useful information from these systems, often in real time. Among these is the estimation of future fuel consumption, which is typically combined with known information about fuel tank capacity and provided to drivers in the form of a Distance-to-Empty (DTE) readout. Accurate prediction of fuel consumption, and thereby DTE, is vital in allowing drivers to know not only when they need to refuel, but also the fuel consumption along different possible routes.

\cite{rodgersetal2013} describes the use of a linear regression technique to predict DTE for electric vehicles. The technique can be adapted for use with internal combustion vehicles to predict their fuel consumption along routes they have driven, based on modeled driving behavior and driving conditions. However, all drivers will not have driven along all possible routes. One driver \(A\) may be interested in finding out how much fuel she is likely to consume along a route \(X\) she has never driven previously. However, if other drivers \(B\) and \(C\) have driven route \(X\), along with other routes \(Y\) and \(Z\) also driven by \(A\), we can use the aforementioned technique to model the differences among the routes, and the differences among the drivers' fuel consumption patterns on the same routes. Combining these models allows us to predict the fuel consumption of driver \(A\) over \(X\).
