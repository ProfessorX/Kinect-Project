
%%% Local Variables: 
%%% mode: latex
%%% TeX-master: t
%%% End: 


\documentclass{sig-alternate}

% Housekeeping
\usepackage{amsmath}
\usepackage{amssymb}

\usepackage{cite}
\usepackage{algorithmic}
\usepackage{array}
\usepackage{graphicx}
\graphicspath{{./}{./Figure/}}

\usepackage{url}





\begin{document}
\title{Improving Building Energy Efficiency by Kinect-based Occupancy
  Tracking and Mobility Detecting System}

% \numberofauthors{1}
% \author{
% \alignauthor
% Haleimah Al Zeyoudi\quad Yanan Xiao\quad Chi-Kin Chau\\
% \affaddr{Masdar Institute of Science and Technology}
% \affaddr{P.O. Box 54224}
% \affaddr{Abu Dhabi, UAE}
% \email{hzeyoudi,yxiao,ckchau@masdar.ac.ae}
% }

\numberofauthors{1}
\author{
\alignauthor
Some author
}


\maketitle{}



\begin{abstract}
  Nowadays, most building air conditioning systems still operate on a
  fixed schedule rather than real-time occupancy. In our study, we
  make an occupancy tracking software based on Kinect to reflect the
  number of people in a open lab. We then build a Markov Chain (MC)
  model after dividing the open area into 4 zones and calculating its
  occupancy respectively. When applying the real-time schedule of one
  week to a building model created with eQuest, we obtain a 22.1\%
  energy reduction in space cooling.
\end{abstract}

% I comment out this part. It seems that it's useless.

% \category{C.3}{Special-purpose and application-based
%   systems}{Real-time and embedded systems}

% \terms{Algorithms, Design, Management}

% \keywords{Wireless sensor networks, Markov chain, Simulation}





% What the heck am I reading.


\section{Introduction}
\label{sec:introduction}
At the core of energy consumption in modern buildings is Heating Ventilation
and Air Conditioning (HVAC) systems which are designed to operate at
full capacity most of the time because it's often assumed a maximum
occupancy. Although current HVAC systems are
equipped with sensors, their management and control systems ignore the dynamic nature
occupancy patterns in buildings. In addition, they are unable to
proactively adjust to occupants' comfort levels. Understanding human mobility
and occupancy patterns are key factors in successfully managing HVAC systems
in buildings. The main contribution of our paper is
to propose an energy-saving model based on occupancy patterns of human
mobility in buildings. The most important features of the system are as follows:

\begin{enumerate}
\item  The \textit{real-time detection and tracking of human mobility}
  in open lab provides accurate occupancy data of an entire floor
  divided into 4 zones.
\item  A \textit{occupancy counting software} carries out the detection,
  tracking and monitoring process based on multiple Microsoft Kinect
  for Windows (K4W) sensors distributed in key locations in the
  building.
\item A \textit{occupancy prediction mechanism} of the building is
  introduced through the use of a MC which models the
  collected occupancy data.  MC is a suitable because it captures the
  temporary nature of occupancy variation along with inter-room
  correlations. Unlike most other building occupancy techniques
  described in~\ref{sec:relatedworks}, our approach is Kinect-based
  and therefore could report live building occupancy states after
  conducting simple calculations, which is also on-line.
\end{enumerate}
% This part should be concised.
\par






\section{Implementation}
\label{sec:implementation}





\section{Results}
\label{sec:results}





\section{Related Works}
\label{sec:related-works}




\section{Conclusion and Future Work}
\label{sec:concl-future-work}










\bibliographystyle{plain}
\bibliography{sigproc}


\end{document}


% End of the day
% Successfully ``manage energy'' in buildings. What the heck.